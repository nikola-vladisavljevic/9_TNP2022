\documentclass{beamer}
\usepackage{beamerthemeshadow}
\usepackage{graphicx}
\usepackage{color}
\usepackage[utf8]{inputenc}
\usepackage{hyperref}
\usepackage[flushleft]{threeparttable}
\usepackage{multirow}
\definecolor{beamer@darkred}{rgb}{0.85,0.1,0.1}
\setbeamercolor{structure}{fg=beamer@darkred}

\def\d{{\fontencoding{T1}\selectfont\dj}}
\def\D{{\fontencoding{T1}\selectfont\DJ}}


\title{Razlika izme\d u programskih jezika C i C++}
\subtitle{-- Seminarski rad iz tehničkog i naučnog pisanja --}
\author{Ognjen Maksimović, Petar Nikolić, Nikola Vladisavljević, Nikola Čuturić}
\institute{Matematički fakultet\\Univerzitet u Beogradu}
\date{
	\footnotesize{Beograd, 2022.}	
}

\begin{document}
\begin{frame}
	\thispagestyle{empty}
	\titlepage
\end{frame}

\addtocounter{framenumber}{-1}

\section{O jeziku C}

\begin{frame}[fragile]\frametitle{O jeziku C}
	\begin{itemize}
		\item Imperativni, proceduralni jezik opšte namene razvijen 1972. godine, namenjen prevashodno pisanju sistemskog softvera.
            \item Nudi prilično direktan pristup memoriji, konstrukcije jezika su tako osmišljene da se jednostavno prevodi na mašinski jezik.
            \item Kreiran u minimalističkom duhu.
	\end{itemize}
\end{frame}

\subsection{Nedostaci jezika C}

\begin{frame}[fragile]\frametitle{Nedostaci jezika C}
	\begin{itemize}
	    \item Dinamička alokacija memorije.
		\item Način na koji funkcionišu pokazivači.
        \item Rad sa niskama
        \item Održivost koda
        \item Kompajleri
	\end{itemize}
        %\frametitle{Pregled} % Table of contents slide, comment this block out to remove it
	%\tableofcontents[hidesubsections] 
\end{frame}

\section{O jeziku C++}

\begin{frame}[fragile]\frametitle{O jeziku C++}
	\begin{itemize}
		\item Najznačajniji direktni naslednik jezika C.
            \item Može se smatrati njegovim objektno-orijentisanim proširenjem.
            \item Započet kao "C sa klasama".
            \item Inspirisan jezicima C, Simula 67, ALGOL 68 i BCPL.
	\end{itemize}
\end{frame}

\section{Razlike izme\d u jezika C i C++}

\begin{frame}[fragile]\frametitle{Razlike izme\d u jezika C i C++}
    \begin{itemize}
	    \item Daleko obimnija standardna biblioteka.
            \item Objektni tipovi (\emph{vector}, \emph{set}, \emph{stack}, \emph{queue}) u standardnoj biblioteci.
            \item Ključna reč \emph{auto}.
            \item Iteratorski tipovi.
            \item Range for petlje.
            \end{itemize}	
\end{frame}

\begin{frame}[fragile]\frametitle{Razlike izme\d u jezika C i C++}
    \centering
    \begin{tabular}{|c|c|c|}
    \hline
    Osobina & C & C++ \\
    \hline
    Programska & \multirow{2}{5.5em}{Imperativna} & Objektno- \\
    paradigma & & -orijentisana\\
    \hline
    Osnova & Asembler & Jezik C \\
    \hline
    Pretprocesorske direktive & Podržane & Podržane\\
    \hline
    Pristup u programiranju & Sa vrha - naniže & Sa dna - naviše\\
    \hline
    \multirow{3}{4em}{Alokacija memorije} & Korišćenjem  & \multirow{3}{5.5em}{Korišćenjem ključnih reči}\\
    & bibliotečkih & \\
    & funkcija &\\
    \hline
    Prenos argumenata & \multirow{2}{5.5em}{Po vrednosti} & Po vrednosti\\
    funkcije & & i po referenci\\
    \hline
    Pokazivači & Podržani & Podržani\\
    \hline
    Upravljanje & \multirow{2}{6.2em}{Nije podržano} & \multirow{2}{4em}{Podržano} \\
    izuzecima & &\\
    \hline
    \end{tabular}
\end{frame}

\section{Gde se danas koriste}

\subsection{Gde se C danas koristi}

\begin{frame}[fragile]\frametitle{Gde se C danas koristi}
    \begin{itemize}
        \item operativni sistemi;
        \item programiranje ugra\d enih sistema;
        \item izrada kompilatora i interpretatora;
        \item u obrazovanju.
    \end{itemize}
\end{frame}

\subsection{Gde se C++ danas koristi}

\begin{frame}[fragile]\frametitle{Gde se C++ danas koristi}
    \begin{itemize}
        \item softver za grafičke obrade (Adobe Photoshop, Adobe Illustrator, itd.);
        \item veliki broj video igara (Unreal engine);
        \item veb pregledači (Mozilla Firefox, Google Chrome);
        \item kompajleri za više programske jezike (Java, C\#, itd.).
    \end{itemize}
\end{frame}

\end{document}