% !TEX encoding = UTF-8 Unicode

\documentclass[a4paper]{article}

\usepackage{color}
\usepackage{url}
\usepackage[T2A]{fontenc} % enable Cyrillic fonts
\usepackage[utf8]{inputenc} % make weird characters work
\usepackage{graphicx}
\usepackage{csquotes}

\usepackage[english,serbian]{babel}
%\usepackage[english,serbianc]{babel} %ukljuciti babel sa ovim opcijama, umesto gornjim, ukoliko se koristi cirilica

\usepackage[unicode]{hyperref}
\hypersetup{colorlinks,citecolor=green,filecolor=green,linkcolor=blue,urlcolor=blue}

%\newtheorem{primer}{Пример}[section] %ćirilični primer
\newtheorem{primer}{Primer}[section]

\begin{document}

\title{Razlika između programskih jezika C i C++\\ \small{Seminarski rad u okviru kursa\\Tehničko i naučno pisanje\\ Matematički fakultet}}

\author{Ime i prezime autora\\ kontakt email adresa autora}
\date{24.~oktobar 2017.}
\maketitle

\abstract{
U ovom tekstu je ukratko prikazana osnovna forma seminarskog rada. Obratite pažnju da je pored ove .pdf datoteke, u prilogu i odgovarajuća .tex datoteka, kao i .bib datoteka korišćena za generisanje literature. Na prvoj strani seminarskog rada su naslov, apstrakt i sadržaj, i to sve mora da stane na prvu stranu! Kako bi Vaš seminarski zadovoljio standarde i očekivanja, koristite uputstva i materijale sa predavanja na temu pisanja seminarskih radova. Ovo je samo šablon koji se odnosi na fizički izgled seminarskog rada (šablon koji \emph{morate} da ispoštujete!) kao i par tehničkih pomoćnih uputstava.}

\tableofcontents

\newpage

\section{Uvod}
\label{sec:uvod}
Uvod

\section{O jeziku C}
Programski jezik C je programski jezik opšte namene koji je 1972. godine razvio Denis Riči\footnote{Dennis Ritchie (1941--2011), američki informatičar}. C je jezik koji je bio namenjen prevashodno pisanju sistemskog softvera i to u okvira operativnog sistema Unix. C je danas prisutan na širokom spektru platformi -- od mikrokontrolera do superračunara.

Jezik C spada u grupu imperativnih, proceduralnih programskih jezika. Kako je izvorno bio namenjen za sistemsko programiranje, programerima nudi prilično direktan pristup memoriji i konstrukcije jezika su tako osmišljene da se jednostavno prevodi na mašinski jezik. Jezik je kreiran u minimalističkom duhu -- ima mali broj ključnih reči, a dodatna funkcionalnost programerima se nudi uglavnom kroz korišćenje bibliotečkih funkcija.

\section{Nedostaci jezika C}
Uprkos svojoj brzini i efikasnosti programski jezik C ima izvestan broj nedostataka, koji je možda najbolje opisan citatom Denisa Ričija, čoveka koji je napravio C: \textquote{C ima moć asemblerskih jezika i koristi se lako kao asemblerski jezici.} Asemblerski jezici su ozloglašeni po težini korišćenja i Denis Riči je ovime hteo da ukaže da je C takođe komplikovan i da ne pruža puno olakšica programerima koji ga koriste.

Neki od poznatijih zamerki na C su: način na koji se dinamički alocira memorija koristeći {\em malloc} i {\em calloc} koji može dovesti do curenja memorije ukoliko se ona ne oslobađa pravilno ili do pristupa delu memorije koji je već oslobođen što bi takođe dovelo do greske. Način na koji funkcionišu pokazivači takođe zna da bude problematičan i njihovo netačno korišćenje od strane programera može dovesti do korupcije memorije. Moguće je i da više pokazivača sadrže istu adresu i time program ne bude maksimalno optimizovan, kao što je slučaj u nekim drugim programskim jezicima. Kompajleri nisu od velike pomoći programeru i moguće je napraviti greške koje se mogu detektovati u fazi kompilacije, ali za to kompilator nije sposoban. Niske se čuvaju kao niz podataka tipa {\em char} sa terminirajućom nulom na kraju i njihovo korišćenje i obrada su veoma komplikovani i zahtevaju manipulaciju memorijom od strane programera. Ne postoji ništa što sprečava programera da piše neodrživ kod i istorijski se to dešavalo toliko često da postoje i takmičenja za pisanje što komplikovanijeg koda u programskom jeziku C. Objektno orijentisano programiranje takođe nije podržano u C-u kao i mnoge druge funkcionalnosti koje olakšavaju pisanje i održavanje koda (npr. introspekcija tipa).

\section{Razlike između jezika C i C++}
Najznačajniji direktni naslednik jezika C je jezik C++ koji se, u trenutku nastanka, mogao smatrati njegovim objektno-orijentisanim proširenjem. Kreirao ga je Bjern Stroustrup\footnote{Bjarne Stroustrup (1950), danski informatičar} 1986. godine. C++ je i dalje jedan od najpopularnijih jezika i koristi se za razvoj zahtevnih aplikacija, s jedne strane zbog svojih objektno-orijentisanih svojstava, a s druge zbog bliske veze sa mašinom, u duhu jezika C.

Neka od svojstava jezika C++ koja ne postoje u jeziku C su
\begin{itemize}
    \item mnoštvo objektnih tipova kao što su {\em vector, set, stack, queue}, implementiranih u standardnoj biblioteci
    \item mogućnost prenosa argumenata funkcije i po referenci
    \item range for petlje
\end{itemize}

\section{Zaključak}
\label{sec:zakljucak}
Zaključak


\addcontentsline{toc}{section}{Literatura}
\appendix

\iffalse
\bibliography{seminarski}
\bibliographystyle{plain}
\fi

\begin{thebibliography}{9}

\bibitem{laski2009software} J. Laski and W. Stanley. \emph{Software Verification and Analysis}. Springer- Verlag, London, 2009.

\bibitem{gcc} Free Software Foundation. GNU gcc, 2013. on-line at: http://gcc. gnu.org/.

\bibitem{haltingproblem} A. M. Turing. \emph{On Computable Numbers, with an application to the Entscheidungsproblem}. Proceedings of the London Mathematical Society, 2(42):230–265, 1936.


\end{thebibliography}


\appendix
\section{Dodatak}
Ovde pišem dodatne stvari, ukoliko za time ima potrebe.
Ovde pišem dodatne stvari, ukoliko za time ima potrebe.
Ovde pišem dodatne stvari, ukoliko za time ima potrebe.
Ovde pišem dodatne stvari, ukoliko za time ima potrebe.
Ovde pišem dodatne stvari, ukoliko za time ima potrebe.


\end{document}
